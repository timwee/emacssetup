\documentclass[twocolumn,10pt]{article}
\usepackage{times}
\usepackage[latin1]{inputenc}
\usepackage{epsfig}
\newcommand{\iffull}{\iftrue}
%\newcommand{\iffull}{\iffalse}

\topmargin -.8 in
\headheight 0pt             
\headsep 0pt                
\headheight 0pt                % Height of box containing running head.
\headsep 0pt                   % Space between running head and text.

%\textheight 9.75 in
\textheight 10.3 in
\textwidth 42pc         % Width of text line.
                        % For two-column mode:
\columnsep 2pc          %    Space between columns
\columnseprule 0pt      %    Width of rule between columns.
\hfuzz 1pt              % Allow some variation in column width, otherwise it's
\parskip 0pt            

 
\makeatletter
\renewcommand{\subsection}{\@startsection
  {subsection}{2}{0mm}{-.1\baselineskip}{-.1\baselineskip}%
  {\normalfont\bfseries}}
\renewcommand{\subsubsection}{\@startsection
  {subsection}{3}{0mm}{-.1\baselineskip}{-.1\baselineskip}%
  {\normalfont\bfseries}}
\makeatother

%\newcommand{\myadjust}{\vspace{-.5ex}}
\newcommand{\myadjust}{\vspace{-.2ex}}
\newcommand{\mykey}[1]{{\small $\langle$\textrm{#1}$\rangle$}}
\newcommand{\spc}{\mbox{\mykey{spc}}}
\newcommand{\vbar}{$\,|\,$}
\newcommand{\opt}{$?$}
\newcommand{\pause}{$\cdot\cdot$}
\newcommand{\cursor}{$|$}
\newcommand{\mousemove}{\textbf{mouse-precise-move}}
\newcommand{\mousecircamove}{\textbf{mouse-circa-move}}
\newcommand{\mousedrag}{\textbf{mouse-\linebreak[3]drag}}
\newcommand{\mouseclick}{\textbf{mouse-click}}

\newcommand{\command}[1]{\textsf{\textup{#1}}}
\newcommand{\cmd}{\sffamily\upshape}
\newcommand{\cat}[1]{\textrm{\textit{#1}}}

\newcommand{\myexample}[1]{\emph{#1}}
\newenvironment{mycenter}
{\begin{trivlist}\item \begin{footnotesize}}
{\end{footnotesize}\end{trivlist}}

\newenvironment{myquote}{\begin{quote}\begin{small}}{\end{small}\end{quote}}
 
\newcommand{\compparameters}{
  \settowidth{\labelwidth}{ST:x}
  \setlength{\leftmargin}{\labelwidth}
  \addtolength{\leftmargin}{1ex}
  \setlength{\itemsep}{0ex}
  \setlength{\parsep}{0ex}
  \setlength{\topsep}{.2ex}}

%   \newcommand{\comparison}[2]{%
%   \begin{list}{\compparameters}{}
%     \item[NLT:\hfill] #1
%     \item[ST:\hfill]  #2
%   \end{list}
%   }

\newcommand{\comparison}[2]{%
NLT: ``#1''. ST: ``#2''.}

%\newcommand{\comparisonthree}[3]{%
%\begin{list}{\compparameters}{}
%  \item[NLT:\hfill] #1
%  \item[ST:\hfill] #2
%  \item[ST+Pred:\hfill] #3
%\end{list}}

\newcommand{\comparisonthree}[3]{%
NLT: ``#1''. ST: ``#2''.  ST+Pred: ``#3''.}

\begin{document}
% \title{ShortTalk: Quick Reference}
%  \author{
%    Nils Klarlund\\
%    \copyright Carnegie Mellon University, August 2004}
% \maketitle
\section*{ShortTalk: Quick Reference\\
\normalsize{Nils Klarlund (\copyright Carnegie Mellon University, 2004)}}


\subsection*{Numerals}
\begin{mycenter}
\begin{tabular}{ll}
\cat{positive}   &= \cmd ane\vbar twain \vbar traio \vbar fairn \vbar faif\\
\cat{negative}   &=  \cmd oon \vbar twoon \vbar truo \vbar foorn \vbar foof\\
\cat{signed}     &=    \cmd \cat{positive} \vbar \cat{negative}\\
\end{tabular}
\end{mycenter} 

\subsection*{Punctuation and special}
\begin{mycenter}
\begin{tabular}[t]{ll}
\cat{letter} =  & \parbox[t]{.75\linewidth}{\cmd 
alpha \vbar
bravo \vbar charlie \vbar delta \vbar echo \vbar foxtrot \vbar
golf \vbar hotel \vbar india \vbar juliett \vbar kilo
\vbar lima \vbar mike \vbar november \vbar oscar \vbar papa \vbar
quebec \vbar romeo \vbar sierra \vbar tango \vbar uniform \vbar victor
\vbar whiskey \vbar x-ray \vbar yankee \vbar zulu}
\end{tabular}
\end{mycenter}
\begin{mycenter}
\begin{tabular}[t]{llll}
\cat{symbol} =  & \cmd clam & \vbar & !\\
& \cmd lat & \vbar  &  \symbol{64} \\
& \cmd numb & \vbar  & \#\\
& \cmd dall\vbar dollar & \vbar  & \$\\
& \cmd per & \vbar  & \%\\
& \cmd crat & \vbar  & \symbol{94}\\
& \cmd amp & \vbar  & \&\\
& \cmd star & \vbar  & *\\
& \cmd laip & \vbar  & (\\
& \cmd rye & \vbar  & )\\
& \cmd plus & \vbar  & +\\
& \cmd nus & \vbar  & -\\
& \cmd eke & \vbar  & =\\
& \cmd bar & \vbar  & \symbol{124}\\
& \cmd lace & \vbar  & \{\\
& \cmd race  & \vbar &  \}\\
& \cmd lack & \vbar  & \symbol{91}\\
\makebox[0ex][l]{\cat{phonetic}=\qquad\makebox[0ex][l]{\cmd \cat{symbol} \vbar \cat{letter} \vbar spooce
\vbar toob \vbar loon}}
\end{tabular} 
\begin{tabular}[t]{llll}
& \cmd rack  & \vbar & \symbol{93}\\
& \cmd slash & \vbar  & /\\
& \cmd beck & \vbar  & \symbol{92}\\
& \cmd till & \vbar  & \symbol{126} \\
& \cmd sem & \vbar  & ;\\
& \cmd col  & \vbar  & :\\
& \cmd cam \vbar comma & \vbar  & ,\\
& \cmd doot & \vbar  & .\\
& \cmd loos & \vbar  & $<$\\
& \cmd groot & \vbar  & $>$\\
& \cmd quest & \vbar  & ?\\
& \cmd score & \vbar  & \_\\
& \cmd hive & \vbar  & -\\
& \cmd sing & \vbar  & '\\
& \cmd quote & \vbar  & "\\
& \cmd bing &   & `\\
\end{tabular} 
\end{mycenter} 
{\small \cat{word} is an English word or a \cat{character command},
which is
\begin{mycenter}   \begin{tabular}{l@{}ll}
    ~& \cat{phonetic}  
(\cat{phonetic} \cat{signed}?)$^+$ 
  \end{tabular} 
\end{mycenter}
Words may be followed by a \cat{common-pos}; in that case, they are
inserted at \cat{common-pos}, not at cursor.  The \cat{phonetic}s
\command{loon} or \command{spooce} can only occur by themselves or
with a \cat{signed}.  \command{toob} can be combined only with
\command{loon} or a \cat{signed}.}

\subsection*{Keys, capitalization, and spacing}
\begin{mycenter}
\begin{tabular}[t]{lll}
\cmd spooce \cat{positive}\opt & \mykey{space} $n$ times\\
\cmd loon \cat{positive}\opt & \mykey{enter} $n$ times \\
\cmd choook \cat{positive}\opt & \mykey{backspace} $n$ times\\
\cmd chaiw \cat{positive}\opt & \mykey{delete}  $n$ times\\
\cmd gloof \cat{positive}\opt & \mykey{left}  $n$ times\\
\cmd graif \cat{positive}\opt & \mykey{right} $n$ times\\
\cmd goop \cat{positive}\opt & \mykey{up} $n$ times\\
\cmd gnaith \cat{positive}\opt &  \mykey{down} $n$ times\\
\cmd toob \cat{positive}  & \mykey{tab} $n$ times\\
\cmd speece & no space \\
\cmd capi & capitalize\\
\cmd coomel & camel, i.e.\ no space and capitalize\\ 
\end{tabular}
\end{mycenter}

\subsection*{Common positions}
\begin{mycenter}
\begin{tabular}{lll}
\cat{common-rng} =& \command{tisk}\vbar & between cursor and mouse\\ 
& \command{tat} & selected region (between cursor and mark)\\
\cat{common-pos}  =& \command{loost} \vbar & where last touched before cursor\\
& \command{lairk}\vbar  & where mark is\\
& \command{hare} \vbar & at cursor \\
& \command{tair} \vbar & at mouse, don't move cursor\\
& \command{gook} & at mouse, move cursor\\
\end{tabular}
\end{mycenter}

\subsection*{Structural designators}
\begin{mycenter}
\begin{tabular}{lll@{\hspace{3em}}l}
\makebox[0ex][l]{\cat{struct-simple} =} &&& \makebox[0ex][l]{contained in line}\\
& \cmd char & \vbar &  character\\
& \cmd stretch& \vbar& whitespace\\
& \cmd ting &  \vbar& non-whitespace\\
& \cmd word&  \vbar& word\\
& \cmd eed&    \vbar& identifier (hyphenated words) \\
& \cmd chunk& \vbar & filenames, email addresses,...\\
& \cmd tier & \vbar & a line (w/o.\ terminator)\\
& \cmd line & & a line (w.\ terminator)\\
\end{tabular}
\end{mycenter}
\begin{mycenter} 
\begin{tabular}{lll@{\hspace{3em}}l}
\makebox[0ex][l]{\cat{struct-complex} =}&&& \makebox[0ex][l]{may span lines}\\
& \cmd term & \vbar &  an identifier or block\\
& \cmd inner & \vbar & the inside of enclosing block\\
& \cmd block & \vbar & an enclosing block\\
& \cmd defi & \vbar & a definition\\
& \cmd senten & \vbar & sentence \\
& \cmd para & \vbar & paragraph \\
& \cmd buffer & & text in buffer\\
\end{tabular}
\end{mycenter}
\begin{mycenter}
\begin{tabular}{lll}
\cat{struct} =& \cat{struct-simple} \vbar
\cat{struct-complex} \quad whole structure\\
\cat{struct-partial}=&(\command{prin} \vbar \command{prex})
\cat{struct} \quad\underline{p}a\underline{r}tial, to beginning (\command{prin})/end (\command{prex}) \\
\cat{struct-rng}= &\cat{struct} \cat{signed} \opt\vbar  \quad 
forwards (\cat{positive})/backwards (\cat{negative})\\
& \cat{struct-partial}\\
\end{tabular}
\end{mycenter}

%\subsection*{Search designators}

\subsubsection*{Search range designators}

\begin{mycenter}
\begin{tabular}{ll@{}ll}
  \cat{search-rng} = & (&\command{rorch} \vbar & backward\\
  &&\command{sorch}) & forward\\
  &\makebox[0pt][l]{\cat{words}} 
\end{tabular}
\end{mycenter}
\myadjust{}

\subsubsection*{Search position designators}

\begin{mycenter}
\begin{tabular}{ll@{}ll}
  \cat{search-pos} = &\makebox[0ex][l]{\command{pen}?} &&penultimate \\
&(& \cmd aift \vbar & after and forward\\
&& \cmd baif \vbar & before and forward\\
&& \cmd ooft \vbar & after and backward\\
&& \cmd boof) & before and backward\\
&\cat{\makebox[0ex][l]{words}}\\
  \cat{pos} = & \makebox[0ex][l]{\cat{common-pos} \vbar
    \cat{search-pos}}\\
\cmd stroop &&& end of argument, neutral word\\
\end{tabular}
\end{mycenter}
\begin{footnotesize}
  Commands, except character commands and \command{speece}, terminate
  argument.
\end{footnotesize}


\subsection*{Cursor commands}

\begin{mycenter}
\begin{tabular}{lll}
\cat{movement} = 
& \cmd nairx \vbar & go to beginning of structures [``next''] and \\
& \cmd noorx \vbar & go to end of structures \\
& \cmd skaip\vbar & to end of structures (forward) \\
& \cmd skoop & to end of structures (backward) \\
\end{tabular}
\end{mycenter}
\begin{mycenter}
\begin{tabular}{lll}
\cat{movement} \cat{struct} \cat{positive}? & move \\
\cat{struct} \cat{positive}? & move to beginning of structure (forward)\\
\cat{struct} \cat{negative}? & move to beginning of structure (backward)\\
\cat{movement} \em{character}
\cat{positive}? & 
move to
character occurrences  \\
\cmd (ghin \vbar ex) \cat{struct} & be\underline{gin}ning or end (\underline{ex}it) of current structure\\ 
\cmd (ghin \vbar ex) block \cat{positive}\opt & leave blocks,
go to beginning or end \\
\cmd go \cat{search-pos} ... & go to first match\\
\cmd mairk & set mark explicitly\\
\cmd gairk \cat{positive}\opt & unravel: go to $n$th last mark\\
\cmd goork & opposite of gairk, go to unravelled mark\\
\cmd goost & go to loost\\
\end{tabular}
\end{mycenter}


\subsection*{Text movement commands}
\label{sec:movement-commands}
\begin{mycenter}
\begin{tabular}{l@{\,}lll}
\cmd grab on\opt \cat{struct-rng} & copy text at mouse to cursor\\
\cmd pull on\opt \cat{struct-rng} & move text at mouse  to cursor\\
\cmd paste on\opt \cat{struct-rng} & copy text at cursor to mouse\\
\cmd push on\opt \cat{struct-rng} & move text at cursor to mouse\\
\cmd swap \cat{struct-rng}?  \cat{common-pos}?
& swap between cursor  and position\\
\cmd trans \cat{struct-rng} & transpose occurrence and following \\
\end{tabular}
\end{mycenter}
\begin{footnotesize}
Note: ``on'' means: replace the single \cat{struct} at destination.
\end{footnotesize}
\subsection*{Text deletion and copy-to-clipboard commands}
\myadjust
\begin{mycenter}
\begin{tabular}{llll}
\cat{del-copy} = & \cmd rem & \vbar & copy and delete (``cut'')\\
& \cmd smack  &\vbar & delete, but no copy (``delete'')\\
& \cmd save &\vbar & copy (``copy'')\\
& \cmd choose  &  & highlight (``select''), no copy
\end{tabular}
\end{mycenter}
\myadjust
\begin{mycenter}
\begin{tabular}{l l}
\cat{del-copy struct-rng} \cat{common-pos}? &  del/copy
range at \cat{pos}\\
\cat{del-copy} (\cat{common-rng} \vbar \cat{search-rng}) &
del/copy range\\
\end{tabular}
\end{mycenter}
\begin{footnotesize}
Note: ``choose tat'' highlights range between cursor and mark.
\end{footnotesize}
\subsection*{Directional deletion}

\begin{mycenter}
  \begin{tabular}{ll}
   \cmd kaill \cat{struct} \cat{positive}\opt \cat{positive} & nearest words forward \\
   \cmd reese \cat{struct} \cat{positive}\opt \cat{positive} & nearest words backward \\
  \end{tabular}
\end{mycenter}

\subsection*{Yank clipboard commands}

\begin{mycenter}
  \begin{tabular}{ll}
   \cmd yank \cat{positive}\opt 
   \cat{common-pos} & insert $n$th entry from clipboard stack\\
   \cmd pop \cat{common-pos} & insert and pop clipboard stack\\ 
 \end{tabular}
\end{mycenter}

\subsection*{Changes}

\begin{mycenter} 
  \begin{tabular}{l@{}l@{}lll}
    ~~~&(& \command{caip}  \cat{struct-rng}? & \vbar & capitalize\\
&& \command{laiw}  \cat{struct-rng}? & \vbar & lowercase \\
&& \command{aipper} \cat{struct-rng}? & \vbar & uppercase \\
&& \command{hive}   \cat{struct-rng}? & \vbar & hyphenate\\
&& \command{speece} & \vbar & delete spaces\\
&& \command{space} & \vbar & normalize to one space\\
&& \command{fix} & \vbar & apply spacing heuristics \\
&& \command{chaiw} & \vbar & delete\\
&& \command{chow} & \vbar & backspace\\
&& \command{join} & \vbar & join line with next\\  
&& \command{lindent} \cat{struct-rng}? & \vbar & left indent range\\
&& \command{rindent} \cat{struct-rng}? & \vbar & right indent range\\
&& \command{comment} \cat{struct-rng}? & \vbar & comment range\\
&& \command{fill} \cat{struct-rng}? &  \vbar & fill \\
&& \command{pound}  \cat{struct-rng}?) & & compound, remove spaces\\
&& \makebox[0ex][l]{%(
\cat{pos} %
% \vbar \cat{struct-boundary})
}
  \end{tabular}
\end{mycenter}
\begin{footnotesize}
  If \cat{struct-rng} is provided, then \cat{pos} is optional with
  default ``hare.''
\end{footnotesize}


\subsection*{Paired delimiter commands}

\begin{mycenter} 
  \begin{tabular}{lllllll}
    \cat{pair-type} = 
    & \command{par} & \vbar & (\ldots)     & \command{quote} & \vbar & \symbol{34} \ldots\symbol{34} \\
    & \command{brace} & \vbar & \{ \ldots \}     & \command{sing} & \vbar & '\ldots'\\
    & \command{bracket} & \vbar & \symbol{91} \ldots \symbol{93} 
    & \command{bing} & & `\ldots` \\
    \cat{pair-type} \command{pair} & \makebox[0ex][l]{insert delimiters, around region
    if possible}\\
    \cat{pair-type} \command{nix} & \makebox[0ex][l]{delete delimiters}\\
    \end{tabular}
  \end{mycenter}

  \subsection*{Window management}
  \begin{mycenter}
    \begin{tabular}[t]{lll}
      \cat{window-op} =& \command{maxi} & maximize window \\
      & \command{mini} & minimize window, choose next \\
      & \command{vaix} & choose other window\\
      & \command{voox} & choose other window, reverse direction\\
      & \command{sploot} & split current window\\
    \end{tabular}
  \end{mycenter}
  \begin{mycenter}
    \begin{tabular}[t]{llll}
      \makebox[0ex][l]{\cat{window-action} =}& \\
      ~~    & \command{menu} & \vbar & show buffer menu\\
      & \command{open} & \vbar & open a file\\
      & \command{save} & \vbar  & save file\\
      & \command{write} & \vbar  & write file (save as) \\
      & \command{insert} & \vbar  & insert file\\
      & \command{kaill} & \vbar  & kill buffer\\
      & \command{direct} & \vbar  & directory listing\\
      & \command{switch} & \vbar  & most other recently shown file\\
      & \command{grow} \cat{positive}?  & \vbar & make window bigger\\
      & \command{shrink}  \cat{positive}? & \vbar & make window smaller \\
      & \command{up} & \vbar & page up\\
      & \command{down} & \vbar & page down\\
      & \command{center} & \vbar & center window\\
      & \command{head} & \vbar &  current line to window head \\
      & \command{bottom} & \vbar &  current line to window bottom\\
      & \cat{positive} \vbar \cat{negative}& &down or up a fraction of page\\
    \end{tabular}
  \end{mycenter}

  \begin{mycenter}
    \begin{tabular}[t]{lll}
      \cat{window-op} \cat{window-action} & do op and action
    \end{tabular}
  \end{mycenter}

  \begin{mycenter}
    \begin{tabular}[t]{lll}
      \command{gaix} \cat{positive}?& go to $n$th next window\\
      \command{goox} \cat{positive}?& go to $n$th previous window\\
    \end{tabular}
  \end{mycenter}

  \begin{mycenter}
    \begin{tabular}[t]{lll}
      \cat{color} =  \command{red} \vbar \command{blue} \vbar \command{green} \vbar
      \command{brown} \vbar  \command{purple} \vbar  \command{orange} \vbar
      \command{yellow} \vbar \command{pink}
    \end{tabular}
  \end{mycenter}

  \begin{mycenter}
    \begin{tabular}[t]{lll}
      \cat{window-op} \cat{color} & do op, select view per color
    \end{tabular}
  \end{mycenter}

  \subsection*{Cross-referencing commands}
  \begin{mycenter}
    \begin{tabular}[t]{lll}
      \cat{cross-ref-cmd} = \cat{window-op} \cat{cross-ref} \cat{common-pos}?
    \end{tabular}
  \end{mycenter}

  \begin{mycenter}
    \begin{tabular}[t]{lll}
      \makebox[0ex][l]{\cat{cross-ref} =} & \\
      ~~    & \command{help}& editor help \\
      &    \command{f-help} & function help (programming languages)\\
      &    \command{v-help} & variable help (programming languages)\\
      &    \command{f-def} & function definition (programming languages)\\
      &    \command{v-def} & variable definition (programming languages)\\
      &    \command{i}& follow indexed reference\\
      &    \command{info} & Emacs info\\
      &    \command{apropos} & Emacs apropos\\
      &    \command{command} \command{apropos} & Emacs command apropos
    \end{tabular}
  \end{mycenter}

  \subsection*{Browsing}
  \begin{mycenter}
    \begin{tabular}[t]{lll}
      \command{browse} \cat{common-pos} & invoke browser on URL at position\\
      \command{browse} \command{direct} & invoke external file browser
    \end{tabular}
  \end{mycenter}

  \subsection*{Repetition of last command}
  \begin{mycenter}
    \begin{tabular}[t]{lll}
      \command{goink} \cat{positive}? & repeat last command a number of times
    \end{tabular}
  \end{mycenter}

  \subsection*{Repetition of command sequence}
  \begin{mycenter}
    \begin{tabular}[t]{lll}
      \command{vox} \command{rec} & record a macro \\
      \command{goink} & stop recording and play \\
      \command{vox} \command{stop}& stop recording \\
      \command{vox} \command{play}& play previous macro \\
    \end{tabular}
  \end{mycenter}


  \subsection*{E-mail support}
  \begin{mycenter}
    \begin{tabular}[t]{ll}
      \command{vox} \command{read} \command{mail}& switch to inbox \\
       \cmd vox send mail& send current message \\
       \cmd vox reply& reply to message \\
       \cmd vox follow-up& reply to all \\
       \cmd vox forward& forward message \\
    \end{tabular}
  \end{mycenter}

  \subsection*{Specialized support}

  \begin{mycenter}
    \begin{tabular}[t]{lll}
      \cmd pleat & dynamic completion\\
      \cmd c-pleat & choose dynamic completion\\
      \cmd vox quit &  listen-quit \\
%      vox rename & rename buffer \\
%      reg replace & query-replace-regexp \\
%      begin comment & begin-comment \\
%      end comment & end-comment \\
     \cmd go line... & goto a specified line number \\
     \cmd meta toob & meta-tab, complete according to mode\\
     \cmd meta X. & meta-X,  execute command \\
     \cmd meta N.& meta-N, next \\
     \cmd meta P.& meta-P, previous \\
    \end{tabular}
  \end{mycenter}

  \subsubsection*{Lisp}
  \begin{mycenter}
    \begin{tabular}[t]{lll}
      \cmd eval express & eval-expression \\
      \cmd eval (\cat{common-pos} \vbar \cat{struct-complex}) & evaluate
      expression \\
      \cmd eval print& eval-print-last-sexp \\
    \end{tabular}
  \end{mycenter}


  \subsubsection*{\LaTeX{}}
  \begin{mycenter}
    \begin{tabular}[t]{lll}
      \command{m-tech}... & insert command\\
      \command{m-no tech} \cat{common-pos}? & delete macro\\
      \command{e-tech}...& insert environment\\
      \command{e-change tech} ... & change environment\\
      \command{e-no tech} \cat{common-pos}? & delete environment
      tags\\
      \command{e-end tech} \cat{common-pos}? & close environment\\
    \end{tabular}
  \end{mycenter}

  \begin{mycenter}
    \begin{tabular}[t]{lll}
      \cat{pair-type} += & \cmd tech-quote   & ``$\cdots$''\\
      & \cmd tech-sing & `$\cdots$' \\
      & \cmd tech-brace & $\{\cdots\}$\\
      & \cmd tech-angle & $\backslash$$<$$\cdots$$\backslash>$ \\
    \end{tabular}
  \end{mycenter}

  \begin{mycenter}
    \begin{tabular}[t]{lll}
      \command{tech label} & reftex-label\\
      \command{tech ref} & reftex-reference\\
      \command{tech cite} & reftex-citation\\
      \command{tech e.m.} & ``$\backslash{}$emph$\{\cdots\}$''\\
      \command{tech ...} & various other common \LaTeX commands\\
    \end{tabular}
  \end{mycenter}

  \subsubsection*{XML}

  \begin{mycenter}
    \begin{tabular}[t]{lll}
      \command{snex}... & insert begin and end tag\\
      \command{dent snex}... & insert indented begin and end tag\\
      \command{snoox}...& insert empty element tag\\
      \command{x-change tag} ... & change tag name\\
      \command{x-end tag} \cat{common-pos}? & close element\\
      \command{x-no tag} \cat{common-pos}? & delete begin
      and end tag\\
      \command{x-quote} & ``\&apos;\vbar\&apos;''\\
      \command{x-angle} & ``\&lt;\vbar\&gt;''\\
      \command{x-cdata} & ``$<$![CDATA[\vbar]]$>$'');\\
    \end{tabular}
  \end{mycenter}

  \subsubsection*{Notation}
  ``?'' means optional, ``+'' means ``one or more''.
%\balancecolumns
\end{document}










